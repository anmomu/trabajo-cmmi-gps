En esta sección se definirá la terminología básica usada en los modelos CMMI \cite{term}.

 
\begin{itemize}
%\item acceptance criteria
%\item acceptance testing
%\item achievement profile
%\item acquirer
%\item acquisition
%\item acquisition strategy
%\item addition
%\item allocated requirement
\item \textbf{Evaluación (\textit{Appraisal}).} Es la examinación de uno o más procesos por un equipo entrenado de profesionales. Para dicha evaluación se utiliza un modelo de referencia de evaluación.
%\item appraisal findings
%\item appraisal participants
%\item appraisal rating
%\item appraisal reference model
%\item appraisal scope
%\item architecture
%\item audit
%\item baseline
%\item base measure
%\item bidirectional traceability
%\item business objectives
%\item capability level
%\item capability level profile
%\item capability maturity model
%\item capable process
%\item causal analysis
%\item change management
\item \textbf{Marco de trabajo CMMI (\textit{CMMI Framework}).} Es la estructura básica que organiza los componentes CMMI, lo que incluye elementos de modelos CMMI actuales, así como reglas y métodos para generar modelos, métodos de evaluación y material de entrenamiento.
\item \textbf{Modelo CMMI (\textit{CMMI model}).} Es un modelo generado del marco de trabajo CMMI.
\item \textbf{\hypertarget{componente}{Componente de modelo CMMI} (\textit{CMMI model component}).} Es cualquier elemento principal de la arquitectura del modelo CMMI, como pueden ser prácticas, objetivos, áreas de proceso, niveles de capacidad o niveles de madurez.
%\item CMMI Product Suite
%\item commercial off-the-shelf
%\item common cause of variation
%\item configuration audit
%\item configuration baseline
%\item configuration control
%\item configuration control board
%\item configuration identification
%\item configuration item
%\item configuration management
%\item configuration status accounting
\item \textbf{Constelación (\textit{Constellation}).} Es una colección de componentes CMMI que son usados para construir modelos, materiales de entrenamiento y documentos relacionados con la evaluación para un área de interés\footnote{Las tres áreas de interés para CMMI son: adquisición, desarrollo y servicios.}.
%\item continuous representation
%\item contractor
%\item contractual requirements
%\item corrective action
%\item customer
%\item customer requirement
%\item data
%\item data management
%\item defect density
%\item defined process
%\item definition of required functionality and quality attributes
%\item deliverable
%\item delivery environment
%\item derived measure
%\item derived requirements
%\item design review
%\item development
%\item document
%\item end user
%\item enterprise
%\item entry criteria
%\item equivalent staging
%\item establish and maintain
%\item example work product
%\item executive
%\item exit criteria
%\item expected CMMI components
%\item findings
%\item formal evaluation process
%\item framework
%\item functional analysis
%\item functional architecture
\item \textbf{\hypertarget{ggoal}{Objetivo genérico} (\textit{Generic goal}).} Es un componente de modelo requerido que describe las características que deben estar presentes para institucionalizar procesos que implementan un área de proceso. 
\item \textbf{\hypertarget{gpractice}{Práctica genérica} (\textit{generic practice}).} Es un componente de modelo esperado que es considerado como importante para la consecución del objetivo genérico que tiene asociado.
%\item generic practice elaboration
%\item hardware engineering
%\item higher level management
%\item incomplete process
%\item informative CMMI components
%\item institutionalization
%\item interface control
%\item lifecycle model
%\item managed process
%\item manager
%\item maturity level
%\item measure (noun)
%\item measurement
%\item measurement result
%\item memorandum of agreement
%\item natural bounds
%\item nondevelopmental item
%\item nontechnical requirements
%\item objectively evaluate
%\item operational concept
%\item operational scenario
%\item organization
%\item organizational maturity
%\item organizational policy
%\item organizational process assets
%\item organization’s business objectives
%\item organization’s measurement repository
%\item organization’s process asset library
%\item organization’s set of standard processes
%\item outsourcing
%\item peer review
%\item performance parameters
%\item performed process
%\item planned process
%\item policy
%\item process
%\item process action plan
%\item process action team
%\item process and technology improvements
%\item process architecture
\item \textbf{\hypertarget{processarea}{Área de proceso} (\textit{Process Area}).} Es un \textit{cluster} de prácticas relacionadas en un área que, cuando se implementan colectivamente, satisface un conjunto de objetivos considerados importantes para la mejora en dicha área. 
%\item process asset
%\item process asset library
%\item process attribute
%\item process capability
%\item process definition
%\item process description
%\item process element
%\item process group
%\item process improvement
%\item process improvement objectives
%\item process improvement plan
%\item process measurement
%\item process owner
%\item process performance
%\item process performance baseline
%\item process performance model
%\item process tailoring
%\item product
%\item product baseline
%\item product component
%\item product component requirements
%\item product lifecycle
%\item product line
%\item product related lifecycle processes
%\item product requirements
%\item product suite
%\item project
%\item project plan
%\item project progress and performance
%\item project startup
%\item prototype
%\item relevant stakeholder
%\item representation
%\item required CMMI components
%\item requirement
%\item requirements analysis
%\item requirements elicitation
%\item requirements management
%\item requirements traceability
%\item return on investment
%\item risk analysis
%\item risk identification
%\item risk management
%\item senior manager
%\item service
%\item service agreement
%\item service catalog
%\item service incident
%\item service level
%\item service level agreement
%\item service level measure
%\item service line
%\item service request
%\item service requirements
%\item service system
%\item service system component
%\item service system consumable
%\item shared vision
%\item software engineering
%\item solicitation
%\item solicitation package
%\item special cause of variation
\item \textbf{\hypertarget{sgoal}{Objetivo específico} (\textit{Specific goal}).} Es un componente de modelo requerido que describe las características únicas que deben estar presentes para satisfacer el área de proceso.
\item \textbf{\hypertarget{spractice}{Práctica específica} (\textit{specific practice}).} Es un componente de modelo esperado que es considerado como importante para la consecución del objetivo específico asociado.
%\item stable process
%\item staged representation
%\item stakeholder
%\item standard (noun)
%\item standard process
%\item statement of work
%\item statistical and other quantitative techniques
%\item statistical process control
%\item statistical techniques
%\item subpractice
%\item subprocess
%\item supplier
%\item supplier agreement
%\item sustainment
%\item system of systems
%\item systems engineering
%\item tailoring
%\item tailoring guidelines
%\item target profile
%\item target staging
%\item team
%\item technical data package
%\item technical performance
%\item technical performance measure
%\item technical requirements
%\item traceability
%\item trade study
\item \textbf{Entrenamiento (\textit{Training}).} Son opciones de aprendizaje, tanto formales como informales.
%\item unit testing
%\item validation
%\item verification
%\item version control
%\item work breakdown structure (WBS)
%\item work group
%\item work plan
%\item work product
%\item work product and task attributes
%\item work startup
\end{itemize}